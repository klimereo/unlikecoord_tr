\section{Overview of the chapter}

This chapter presents a formal analysis of Turkish unlike coordination phenomena within the framework of Lexical-Functional Grammar (LFG). In \S\ref{sec:frameworksofsyntax}, the rationale for adopting this particular theoretical framework is justified. Next, in \S\ref{sec:lfg}, the fundamental properties of LFG are introduced, providing essential background information for the analysis. \S\ref{sec:lfganalysis} presents a detailed formal analysis of Turkish unlike coordination within the context of distinct syntactic configurations. Finally, \S\ref{sec:exampleanalysis} demonstrates how the analysis works within the context actual examples of unlike coordination.

\section{Frameworks of syntax}\label{sec:frameworksofsyntax} 

Modern approaches to natural language syntax can be broadly classified into two overarching frameworks \citep{pullumetal_2001}: \textit{generative-enumerative syntax} and \textit{model-theoretic syntax}. Although these two frameworks share the common goal of capturing the syntactic parameters of natural languages, they differ significantly in their approaches to achieving this objective.

\textit{Generative-enumerative syntax} (henceforth abbreviated as GES) has its origins in the syntactic component of mathematical logic and formal language theory, with foundational contributions made by \citet{Chomsky1959}. Within the GES framework, language is viewed as a (potentially infinite) collection of strings generated by a formal ``device.'' This formal device consists of a finite set of rewrite rules, often referred to as phrase-structure rules, and lexical items. Consequently, the responsibility of this device is to generate (and enumerate) all and only the grammatical strings of a given language. Accordingly, linguists working within the GES framework endeavor to formulate a precise set of rules for this device that can adequately account for grammatical strings while avoiding the generation of ungrammatical ones. Prominent theoretical approaches following GES include Minimalism \citep{Chomsky1993} and its precursor, Government and Binding Theory \citep{Chomsky1981}, as well as Tree-Adjoining Grammars \citep{Joshi1987} and various forms of categorial grammars.

In contrast to GES, \textit{model-theoretic syntax} (henceforth abbreviated MTS) primarily originates from the semantic domain of mathematical logic. Unlike GES, MTS does not rely on rewrite rules but rather employs a set of constraints to define syntax. These constraints do not aim to exhaustively enumerate all and only the grammatical strings of a given language. Instead, they collectively define syntax by specifying the necessary conditions for the desired syntactic structure of a valid natural language string. Consequently, a natural language string is deemed grammatical if it satisfies the constraints formulated by a linguist, while it is considered ungrammatical if it violates them. It is important to note that these constraints do not generate syntactically well-formed strings, neither conceptually nor formally. Rather, they model possible syntactic structures that can characterize a well-formed string.  Notable MTS-based approaches include Head-Driven Phrase Structure Grammar (HPSG; \citealp{pollardetal1994, muller:2021}) and Lexical-Functional Grammar (LFG; \citealp{kaplan&bresnan_1982, Dalrymple2019}).

MTS-based approaches offer several advantages over their GES-based counterparts both in formal and empirical domains (for an overview, see \citealp{pullumetal_2001, Sag2011}). One notable advantage of MTS-based approaches lies in their ability to account for gradience in ungrammaticality, a feature lacking in GES-based approaches. In GES-based approaches, grammars generate all and only the grammatical strings that form a set. Thus, a given sentence either belongs to this set (grammatical) or not (ungrammatical). However, this binary classification fails to capture the varying degrees to which a sentence violates syntactic rules. To exemplify this issue, consider the following sentences in (\ref{tonysoprano}). 

\pex[glspace=!1em,everygla={},everyglb={},aboveglbskip=-.15ex, interpartskip=5pt]
\label{tonysoprano}
\a \label{tonysopranoa}A wrong decision is better than indecision.
\a \label{tonysopranob}\ljudge{*} A wrong decision are better than indecision.
\a \label{tonysopranoc}\ljudge{**} Decision a better wrong is indecision than.
\xe

In a GES-based grammar of English, both (\ref{tonysopranob}) and (\ref{tonysopranoc}) would be classified as ungrammatical since they do not belong to the generated set of grammatical strings. While both sentences are indeed ungrammatical, the extent to which they violate the syntactic rules of English differs. Sentence (\ref{tonysopranob}) still retains some ability to express a proposition, albeit with a subject-verb agreement error. In contrast, sentence (\ref{tonysopranoc}) fails to convey any coherent meaning at all. However, within the GES framework, there is no inherent mechanism to account for gradience in ungrammaticality, as grammaticality is treated as a binary property tied to set membership.

On the other hand, an MTS-based grammar provides the necessary tools to model degrees of ungrammaticality. A given string does not need to violate all syntactic constraints simultaneously. For instance, sentence (\ref{tonysopranob}) violates only the constraint that governs the number and person agreement between subject and verb in English, while satisfying all other syntactic constraints. In contrast, sentence (\ref{tonysopranoc}) violates numerous constraints of English syntax. Consequently, an MTS-based grammar can effectively explain and measure the degrees of grammaticality by assessing the number of constraint violations, rather than relying solely on set membership.

Another advantage of MTS stems from its ability to effectively model the incremental processing of language, a crucial aspect of real-life language comprehension.  When individuals encounter a sentence, they begin constructing its meaning and syntactic structure as soon as they start perceiving the linguistic input, without waiting for the full utterance to conclude. This incremental processing involves constant revision and updating of the partial structure in their minds as new linguistic items unfold.\footnote{This phenomenon of incremental processing has been extensively supported by a wealth of psycholinguistic research. See \citet{Levelt1993}, \citet{Tanenhaus1995}, and \citet{VONDERMALSBURG2011}.} 

To effectively model this dynamic and incremental process, a formal mechanism is required to construct and evaluate partial structures. GES-based approaches, however, fall short in this regard since their rewrite rules solely enumerate complete expressions and exclude the fragments that constitute them. For instance, the string \textit{a wrong decision is} does not exist within the set of grammatical strings in GES-based grammars. Consequently, it is treated no differently than a randomly ordered sequence of symbols. 

In contrast, MTS-based approaches, such as HPSG and LFG, are equipped to handle partial structures. In these approaches, the string \textit{a wrong decision} is analyzable and already satisfies numerous constraints pertaining to the structure of English NPs and subject-verb agreement, among others. Moreover, lexical items in MTS-based approaches inherently carry diverse morphosyntactic information regardless of whether they are part of a larger syntactic construction. This means that even as isolated expressions, lexical items contribute to (partial) meaning and morphosyntactic structure. This feature-rich nature of lexical items enables MTS-based approaches to account for partial expressions giving rise to both meaning and morphosyntactic structure. Consequently, it can be claimed that MTS-based approaches enable linguists to more accurately model the intricacies of linguistic knowledge in humans.

Building upon the advantages offered by MTS foundations, this thesis proposes a formal analysis of Turkish coordination within an MTS-based framework, namely Lexical-Functional Grammar (LFG). While LFG also incorporates elements from GES, the analysis presented here integrates mechanisms proposed in previous works to align LFG more closely with the core principles of MTS. Thus, the analysis not only provides a formalized account of the empirical findings regarding Turkish coordination but also exemplifies a form of LFG that more closely adheres to the core principles of MTS.
