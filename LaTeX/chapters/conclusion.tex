\begin{sloppypar}
To the best of this author's knowledge, this thesis constitutes the first attempt at investigating unlike coordination phenomena in Turkish. To maximize scientific validity, two widely employed methodologies in linguistics were utilized: corpus study and acceptability judgment study. Furthermore, these methodologies were employed in a way that complemented each other.
\end{sloppypar}

The results from both the corpus study and the acceptability judgment task provide compelling evidence that Turkish indeed allows for coordination of unlike elements, both in terms of unlike category and unlike case coordination. However, unlike coordination in Turkish is subject to certain restrictions; not all elements can be coordinated with one another. The grammatical function of the conjuncts appears to play a significant role in determining the acceptability of such structures, as hypothesized at the outset of the thesis. This hypothesis finds strong support both in the corpus study and acceptability judgment task. In the corpus study, the vast majority of examples feature conjuncts with matching functions. Additionally, the linear-mixed effects models applied to the acceptability judgment task data further confirm this observation, as mismatching functions were found to be a negative predictor of acceptability ratings. 


In conclusion, the empirical investigations confirm the hypothesis put forward in this research, also providing support for the increasingly prevalent view that coordination does not universally require conjuncts to match in their morphosyntactic properties. However, further empirical data from other languages would indeed be valuable in strengthening this position.

To analyze and formalize the observed empirical facts, the thesis adopts the framework of Lexical-Functional Grammar and follows the approach proposed by \citet{prz:pat:21:oup}. The analysis refrains from encoding an overarching principle that only conjuncts with like functions can be coordinated, as was suggested by \citet{peterson2004}. Instead, it proposes that each conjunct must meet certain constraints within the syntactic position they occupy, which is a position aligned with \citet{prze:22:cases}. Consequently, this approach elegantly accounts for the empirical observations on Turkish coordination without postulating Turkish- or coordination-specific parameters.

Furthermore, the implementation chapter demonstrates that the proposed analysis of unlike coordination can be implemented. The logic behind the implementation was determined to be straightforward and fully compatible with XLE. However, a noteworthy challenge that requires further investigation involves integrating an approach that transfers syntactic categories from c-structure to f-structure within existing XLE implementations and grammar engineering frameworks. 