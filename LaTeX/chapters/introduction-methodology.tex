\section{Methodological approach}

The present thesis aligns itself with the empirically-oriented part of theoretical linguistics, where the sole judgment of the linguist is no longer accepted as a valid source of information. Accordingly, the hypothesis asserted in this thesis is tested against two types of empirical evidence obtained under two distinct methodological approaches: a corpus analysis and an acceptability judgment experiment. 

\begin{sloppypar}
While both types of methodologies are commonly employed in \mbox{empirically-oriented} works, corpus-based methodologies are much more prevalent due to understandable reasons. Corpus-based methodologies are not only easier to deploy, but also better suited for quantitative analyses. Despite these advantages, corpora should not be regarded as the ultimate sources of linguistic evidence (for an alternative view, see \citealp{sampson-2002}, pp.\ 1--13). As \citet[p.\ 29]{sprouse_schutze_judgmentdata} note, the existence (or non-existence) of a specific syntactic construction in a corpus does not constitute ultimate evidence proving/disproving a hypothesis on that construction. After all, if language is acknowledged as inherently a generative and compositional system, then no corpus will ever be able to store all the conceivable sentences of a given language. In the same vein, there is no guarantee that a corpus will contain only the valid sentences of a language, as automatically compiled and spoken-language corpora tend to contain ill-formed sentences. 

This necessitates supplementing the corpus-based methodology with acceptability judgment data, where hypotheses are additionally validated by native speaker judgments. This integration does not only enable the researcher to validate examples discovered in corpora but also test rare constructions that fail to appear in them. Just like corpus-based methods, acceptability judgment data, regardless of the formality of the setting, also suffers from various problems due to judgment data being a type of self-report data. In this respect, the variability between speaker responses \citep[pp.\ 3--7]{cowart1997} seems to be a major concern. Nevertheless, the present thesis regards both corpus data and acceptability judgment data as complementary sources of evidence that should be jointly employed to help balance out their respective shortcomings. In this connection, the thesis also rejects the claim that a single data collection method can reliably and accurately capture the notoriously complex domain of natural languages.
\end{sloppypar}