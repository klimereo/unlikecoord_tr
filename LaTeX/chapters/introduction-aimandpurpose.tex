\section{Background}
\justifying

\textit{Coordination} refers to a grammatical structure wherein two or more linguistic elements are conjoined, typically through coordinating elements known as \textit{conjunctions}. These conjoined elements, referred to as \textit{conjuncts}, can be single words, phrases, or even complete sentences. Despite the ubiquity of coordination in natural languages, the syntactic and semantic characteristics of coordinate structures remain a topic of extensive debate within theoretical linguistics.
Within the context of the present thesis, the central aim is to undertake an empirical investigation and formal analysis of an atypical form of coordination, termed ``unlike coordination,'' in Turkish data. 

\begin{sloppypar}
Unlike coordination occurs when the coordinated elements do not align either in terms of their syntactic categories (i.e., ``coordination of unlike categories'') or in their grammatical cases (i.e., ``coordination of unlike cases''). Illustrative examples of unlike coordination can be seen in the examples in (\ref{ex0}). In the English sentence in (\ref{ex0a}), the elements conjoined by \textit{and} differ in their syntactic category since the first conjunct, \textit{a Republican}, is a noun phrase (NP) while the second conjunct, \textit{proud of it}, is an adjectival phrase (AdjP). The Polish example in (\ref{ex0b}), on the other hand, exemplifies unlike case coordination. In this instance, both conjuncts are NPs, but they differ with respect to the grammatical cases that they bear.
\end{sloppypar}

\pex[glspace=!1em,everygla={},everyglb={},aboveglbskip=-.15ex, interpartskip=15pt]
\label{ex0}
\a\label{ex0a}Pat is [a Republican]\textsubscript{NP} and [proud of it]\textsubscript{AdjP}. \trailingcitation{\citep[p.\ 117, ex.\ (2b)]{sagetal1985}}
\a\label{ex0b} \begingl
\gla Dajcie wina i ca\l{}\k{a} \'swini\k{e}!  //
\glb give wine.\textsc{gen} and whole.\textsc{acc} pig.\textsc{acc} //
\glft `Serve (some) wine and a whole pig!' \trailingcitation{\citep[p.\ 175, ex.\ (5.269)]{prze:99}} //
\endgl
\xe

It should be noted that unlike coordination is not exempt from constraints. Not all configurations of unlike coordination are valid, as exemplified by the ungrammatical sentence in (\ref{ex1c}). While each of the conjuncts in (\ref{ex1c}) retains grammaticality when considered individually in (\ref{ex1a}) and (\ref{ex1b}), respectively; their coordination, which is an unlike category coordination (prepositional phrase (PP) \& complementizer phrase (CP)), yields an ungrammatical sentence. 

\pex
\label{ex1}
\a\label{ex1a}The scene [of the movie]\textsubscript{PP} was in Chicago.
\a\label{ex1b}The scene [that I wrote]\textsubscript{CP} was in Chicago. \trailingcitation{\citep[p.\ 36, ex.\ (24)]{Chomsky57}} 
\a\label{ex1c}\ljudge{*}The scene [of the movie]\textsubscript{PP} and [that I wrote]\textsubscript{CP} was in Chicago.
\trailingcitation{\citep[p.\ 36, ex.\ (25)]{Chomsky57}} 
\xe

\subsection{The law of the coordination of likes}
Unlike coordination (both of categories and cases) has been widely assumed to be impossible since as early as \citet[pp.\ 35--36]{Chomsky57}, where he concludes, on the basis of the examples in (\ref{ex1}), that conjuncts must not only qualify as constituents, as in (\ref{ex1a}) and (\ref{ex1b}), but also have the same category, as evinced by the ungrammaticality of (\ref{ex1c}). This generalization, which is commonly called the Law of the Coordination of the Likes (LCL; \citealp{Wiliams1981}), proved to be one of the well-entrenched views on the syntax of coordination even though contrary evidence started to appear as early as \citet{dik1968} and \citet{peterson1981}. Although the influence of this view is most pronounced in the early research on coordination within the transformational tradition \citep{schachter1977, Wiliams1981}, it still garners support from some researchers.\footnote{A recent defense of the LCL can be found in \citet{Bruening2020}.}

One of the commonly adopted strategies to maintain LCL in the face of contrary data has been to assume an intermediary elliptical process. This approach suggests that unlike coordination arises through the omission of specific elements from the underlying like coordination present in the base-generated sentence. Following this assumption, the unlike category coordination in (\ref{ex2b}), \textit{a plumber and making a fortune}, is a result of the elision of the repeated element from the second conjunct in the underlying like coordination in (\ref{ex2a}), which is the verb \textit{be}.

\pex 
\label{ex2}
\a\label{ex2a}Bill could be a plumber and be making a fortune.
\a\label{ex2b}Bill could be a plumber and $\varnothing$ making a fortune. \trailingcitation{\citep[p.\ 54, ex.\ (12a)]{beavers-sag:2004}} 
\xe

One far-reaching consequence of this analysis is that it renders the surface unlike coordination in (\ref{ex2b}), \textit{a plumber and making a fortune}, a non-constituent. In other words, there is no superior node that governs the entire coordinate structure. \citet{peterson2004}, however, refutes this consequence of ellipsis-based analysis by demonstrating that the surface unlike coordination must be regarded as a constituent based on some widely assumed tests of constituency. These tests include topicalization, as illustrated in (\ref{ex3a}), pro-form \textit{so} substitution, as seen in (\ref{ex3b}), and right node raising, as exemplified in (\ref{ex3c}). Therefore, the phenomenon of unlike coordination cannot be explained away by adopting an ellipsis-based analysis.

\pex
\label{ex3}
\a\label{ex3a}A plumber and making a fortune though Bill may be, he's not
going to be invited to my party. \trailingcitation{\citep[p.\ 648, ex.\ (10a)]{peterson2004}}
\a\label{ex3b}Bill is a plumber and making a fortune, and so is John. \trailingcitation{\citep[p.\ 648, ex.\ (11a)]{peterson2004}}
\a\label{ex3c}Bill is, and John soon will be, a master plumber and making a
fortune. \trailingcitation{\citep[p.\ 649, ex.\ (12a)]{peterson2004}}
\xe

\subsection{Alternative accounts of unlike coordination}
In non-transformational frameworks, such as GPSG\footnote{Generalized Phrase Structure Grammar; \citealp{gazdaretal1985}}, HPSG\footnote{Head-Driven Phrase Structure Grammar; \citealp{pollardetal1994, muller:2021}} and LFG\footnote{Lexical-Functional Grammar; \citealp{kaplan&bresnan_1982, Dalrymple2019}}, unlike coordination has been acknowledged as a genuine case of coordination, and various analyses to account for its behavior have been offered.

\subsubsection{Sag et al. (1985)}
\citeauthor{sagetal1985}'s \citeyearpar{sagetal1985} GPSG analysis of unlike coordination constituted one of the first systematic attempts at formally modeling the phenomenon in question. In their analysis, the sentence (\ref{ex4a}) is grammatical because the individual conjuncts meet the only requirement set by the verb \textsc{become}: they are predicative arguments of the verb. Since \textsc{become} does not impose any constraint regarding the syntactic category of its predicative argument, unlike category coordination becomes possible as long as the individual conjuncts fulfill the external requirements imposed on them. In this line of reasoning, the ungrammaticality of (\ref{ex4b}) can also be explained since the individual conjuncts fail to satisfy the more specific constraints imposed by \textsc{become}. Namely, the conjuncts in the argument position should not only be predicative elements but also either nominals or adjectivals. This grammatical relation between the verb and its coordinated arguments is established through the mother node immediately dominating the conjuncts, which bears only the features that are common to all conjuncts.\footnote{The formal details behind \citeauthor{sagetal1985}'s GPSG analysis are grossly simplified here, as a detailed treatment of their analysis is beyond the scope of the present chapter.} 

\pex
\label{ex4}
\a\label{ex4a}Pat became a Republican and quite conservative. \trailingcitation{\citep[p.\ 142, ex.\ (67a)]{sagetal1985}}
\a\label{ex4b}\ljudge{*}Tracy has become a Republican and of the opinion that we
must place nuclear weapons in Europe. \trailingcitation{\citep[p.\ 142, ex.\ (67b)]{sagetal1985}}
\xe

\subsubsection{Bayer (1996)}
\citeauthor{sagetal1985}'s (\citeyear{sagetal1985}) work demonstrated that there is no intrinsic categorial constraint imposed upon the conjuncts by the coordination itself. The constraints on coordinate structures can instead be explained and modeled in terms of more fundamental syntactic relations between the coordination and its external governor. \citeauthor{sagetal1985}'s (\citeyear{sagetal1985}) analysis, however, was driven primarily by English data and, as a consequence, effectively only dealt with unlike category coordination in complement position. 

The analysis proposed by \citet{bayer1996} frames the phenomenon solely as a relationship between individual conjuncts and their external governor. Under his analysis, syntactic constraints are encoded strictly in terms of disjunctive or conjunctive statements that are, in effect, evaluated for each conjunct. For instance, \textsc{become} specifies that its predicative complement must satisfy the disjunctive statement NP $\lor$ AP, which, in the case of coordination, is evaluated for the category of each conjunct. This simple mechanism predicts that among the two sentences represented in (\ref{ex4}), (\ref{ex4a}) is the valid one and not the (\ref{ex4b}).

\subsubsection{Peterson (2004)}
One of the first detailed LFG treatments of unlike coordination can be found in \citet{peterson2004}, where (un-)gram\-maticality of unlike coordination is shown to be a natural consequence of the built-in formal features of LFG for modeling coordination phenomena. According to Peterson, the parameters of unlike coordination do not hinge upon syntactic category considerations but instead upon the grammatical functions of individual conjuncts. In his LFG analysis, the sentence (\ref{ex5a}) is grammatical because the particular model of coordination he offers only allows conjuncts with matching functions to connect to their external dependents. In this example, since both conjuncts, \textit{awake} and \textit{asking for you}, have predicative functions, they can be associated with their subject, \textit{the children}. If the conjuncts have different grammatical functions, as in (\ref{ex5b}), where we have a coordination of a direct object, \textit{a unicorn}, and a predicate, \textit{happy}, \textit{happy} cannot be associated with its intended subject, \textit{John}.

\pex
\label{ex5}
\a\label{ex5a}The children are awake and asking for you. \trailingcitation{\citep[p.\ 664, ex.\ (45)]{peterson2004}}
\a\label{ex5b}\ljudge{*}John saw a unicorn and happy. \trailingcitation{\citep[p.\ 664, ex.\ (46b)]{peterson2004}}
\xe

\subsubsection{Patejuk (2015) and Dalrymple (2017)}
Peterson's strong claim regarding the functional equivalence requirement between the conjuncts has been questioned in the LFG literature. First, \citet[ch.\ 5]{pat:15} points out that cross-linguistic data, primarily from Russian \citep{chaves-paperno:2007} and Polish \citep[][pp.\ 2200--2202]{pat:prz:12b}, show that different grammatical functions can be coordinated in certain syntactic contexts.\footnote{See \citet[ch.5]{pat:15} for a detailed catalog of papers dealing with the coordination of unlike grammatical functions in various European languages.} Second, \citet{Dalrymple2017} draws attention to the fact that Peterson's phrase-structure rule for coordination (see (\ref{ex6})), where the category of the coordination is equated with the category of the first conjunct, allows unlike coordination so long as the category of the first conjunct is compatible with the external categorial requirements imposed on the coordination.

\pex
\label{ex6}
X \hspace{5pt} $\longrightarrow$ \hspace{5pt} X \hspace{5pt} Conj \hspace{5pt} Y \trailingcitation{\citep[p.\ 652, ex.\ (20)]{peterson2004}}
\xe

\noindent As a consequence, Peterson's coordination rule represented in (\ref{ex6}) classifies the ungrammatical coordinate structure in (\ref{ex5b}) as a licit one because the category of the coordination (NP), derived from the category of the first conjunct, is compatible with the category requirement imposed by \textsc{see}. In order to deal with these issues, \citet{pat:15} introduces a new syntactic category label, termed UP, for unlike category coordination. 

\begin{sloppypar}
\citeauthor{Dalrymple2017}'s (\citeyear{Dalrymple2017}) account regards Patejuk's UP proposal as insufficient and instead replaces atomic syntactic categories with feature matrices where, for instance, the atomic categories of N and NP are uniformly represented as \mbox{[\textsc{n} +, \textsc{v} --, \textsc{p} --, \textsc{adj} --, \textsc{adv} --]}. As for coordination, the feature matrix associated with the coordination node is understood as a ``collection hub'' where categorial information present in individual conjuncts is aggregated. For example, the feature matrix of an unlike coordination conjoining a nominal and an adjectival element would be encoded as [\textsc{n} +, \textsc{adj} +]. 
\end{sloppypar}

Complex categorial constraints imposed by governors are also encoded in terms of these feature matrices. So, when a governor constrains the category of its arguments, it states what categories are \textit{not} allowed and leaves the co-occurrence of allowed categories as a viable possibility. For instance, the sentence (\ref{ex9a}) is grammatical because the category of the coordination, [\textsc{n} +, \textsc{adj} +], does not violate the categorial constraint imposed by \textsc{become}, which is [\textsc{v} --, \textsc{p} --, \textsc{adv} --]. The same mechanism predicts the ungrammaticality of the sentence (\ref{ex9b}) because the category of coordination, [\textsc{n} +, \textsc{p} +], clashes with the constraints imposed by \textsc{become}.\footnote{Like other analyses presented in this chapter, this is a highly simplified account of \citeauthor{Dalrymple2017}'s (\citeyear{Dalrymple2017}) analysis.}


\pex
\label{ex9}
\a\label{ex9a}Fred became a professor and proud of his work. \trailingcitation{\citep[p.\ 35, ex.\ (6a)]{Dalrymple2017}}
\a\label{ex9b}\ljudge{*}Fred became a professor and in line for a promotion. \trailingcitation{\citep[p.\ 35, ex.\ (6c)]{Dalrymple2017}}
\xe

\subsubsection{Przepiórkowski and Patejuk (2021)}
Nevertheless, based on cross-linguistic data found in corpora, \citet{prz:pat:21:oup} question Dalrymple's analysis by pointing out that the constraints imposed by governors are usually not limited to basic syntactic categories but extend to specific references to complementizers, prepositions, and grammatical cases as well.

Consider the English verb \textit{wait}. Its argument position does not indiscriminately accommodate any PP or CP; rather, it specifically necessitates a PP headed by \textit{for} or a CP headed by \textit{until}. This is evident from the observation that substituting other prepositions or complementizers for \textit{for} or \textit{until} in a sentence containing \textit{wait} results in ungrammaticality. Moreover, this constraint also applies to coordinated arguments of \textit{wait}. In (\ref{ex10}), we observe that the argument position of \textit{wait} features a coordination of unlike categories (PP \& CP). Nonetheless, this construction is grammatical since each conjunct individually satisfies the complex morphosyntactic requirements imposed by \textit{wait} on its arguments.

\pex[glspace=!1em,everygla={},everyglb={},aboveglbskip=-.15ex, interpartskip=15pt]
\label{ex10}
Next time you’re thinking about seeing that upcoming sequel, consider
waiting [for the DVD]\textsubscript{PP[\textit{for}]} or [until it’s out on Netflix]\textsubscript{CP[\textit{until}]}. \trailingcitation{\citep[p.\ 211, , ex.\ (14)]{prz:pat:21:oup}}
\xe

\begin{sloppypar}
Consequently, \citeauthor{prz:pat:21:oup} (pp.\ 216--220) propose an LFG-based solution, wherein both categorical and other functional constraints, such as case and preposition form, are handled at the level of individual conjuncts through constraining statements that do not refer to the features carried by the mother node. Their analysis effectively characterizes unlike category coordination as a structure devoid of any specific syntactic category and models syntactic category as a functional rather than a strictly phrase-structural term. Their approach, which is akin to Bayer's, serves as the foundation for the formal analysis of Turkish unlike coordination in the context of the present thesis.
\end{sloppypar}

\subsubsection{Przepiórkowski (2022)}

\begin{sloppypar}
Apart from syntactic category considerations, in morphosyntactically more complex languages, the possibility of conjoining nominals bearing different grammatical cases enters into the equation as well (i.e., unlike case coordination), which can be observed in the Polish examples in (\ref{ex11}). 
\end{sloppypar}

\pex[glspace=!1em,everygla={},everyglb={},aboveglbskip=-.15ex, interpartskip=15pt]
\label{ex11}
\a \label{ex11a}
\begingl
\gla Przyjedzie albo p\'{o}\'{z}nym wieczorem, albo nast\k{e}pnej zimy.  //
\glb come.\textsc{fut}.\textsc{3sg} or late.\textsc{ins}.\textsc{sg}.\textsc{m} evening.\textsc{ins}.\textsc{sg}.\textsc{m} or next.\textsc{gen}.\textsc{sg}.\textsc{f} winter.\textsc{gen}.\textsc{sg}.\textsc{f}//
\glft `(S)he will come either late in the evening, or next winter' \trailingcitation{\citep[p.\ 175, ex.\ (5.270)]{prze:99}} //
\endgl
\a \label{ex11b}
\begingl
\gla Ja i trzech innych nosimy j\k{a} w lektyce  ...  //
\glb I.\textsc{nom}.\textsc{sg} and three.\textsc{acc}.\textsc{pl}.\textsc{m} others.\textsc{gen}.\textsc{pl}.\textsc{m} carry.\textsc{1pl} she.\textsc{acc} in litter {}{}{}//
\glft `Me and three others are carrying her in a litter  . . .' \trailingcitation{\citep[p.\ 607, ex.\ (36)]{prze:22:cases}} //
\endgl
\xe

\begin{sloppypar}
In the literature on unlike coordination, there has been a greater focus on unlike category coordination compared to unlike case coordination. This discrepancy could be attributed to the extensive discussion of unlike coordination being based primarily on English, a language that practically lacks a system of grammatical case.  However, a recent paper by \citet{weisser2020} proposes a cross-linguistic generalization called \textit{Symmetry of Case in Conjunction} (p.\ 43). According to Weisser, the grammatical cases of the conjuncts always match at an underlying level, even in apparent mismatches. Weisser suggests that these apparent mismatches can be explained through ellipsis or other superficial morphological processes.

\citet{prze:22:cases} argues against Weisser's generalization and presents genuine instances of unlike case coordination primarily from Polish. Przepi\'{o}rkowski asserts that these instances cannot be explained through ellipsis or other superficial morphological operations, leading to the conclusion that there is no cross-linguistic requirement for conjuncts to bear the same grammatical case. While Przepi\'{o}rkowski's data indeed call into question the \textit{Symmetry of Case in Conjunction} generalization, further cross-linguistic data is needed to draw more robust conclusions about the parameters governing the grammatical cases of conjuncts.
\end{sloppypar}

\section{Aim and scope of the thesis}
\subsection{Problems and deficits}
Despite numerous analyses of unlike coordination proposed over the past four decades, the study of coordination of unlikes remains active and the subject continues to be a matter of contention \citep[p. 593]{prze:22:cases}. Nevertheless, the progress of the subject seems to be hindered by two distinct empirical challenges: 1) a lack of diverse linguistic data; 2) a lack of proper experimental approaches to data validation. 

The importance of the first challenge can be easily appreciated when we take into account the fact that the emergence of new data has consistently played a pivotal role in driving further insights and refinements within the literature.  Yet, the current discussion on unlike coordination is still exclusively driven by data from English and Polish. The second challenge, however, is more controversial, as there does not seem to be a clear consensus among researchers regarding the adoption of formal experimental methodologies in syntax research.\footnote{For arguments in favor of formal experimentation methods, see \citet{Schutze1996}, \citet{cowart1997}, \citet{WASOW2005}, \citet{gibsonetal13}, and \citet{HitzFrancis+2016}. For arguments that cast doubt upon their necessity, see \citet{phillips2009should} and \citet{SPROUSE2013}.} Since the present thesis is categorically situated within the field of cognitive science and regards linguistics as one of the cognitive sciences, it employs formal experimentation methods to study the phenomena that constitute the research object.

\subsection{Hypothesis statement and the overview of the thesis}
In light of these issues, the present thesis attempts to further refute the assumed coordination of likes constraint based on novel empirical and experimental data from Turkish. Consequently, the primary hypothesis put forward in this thesis is that conjuncts may differ both in their syntactic categories and cases in Turkish as long as they match in their grammatical functions. 

\begin{sloppypar}
While there is no universally agreed-upon set of grammatical functions, the present work acknowledges a subset of grammatical functions commonly assumed in nontransformational and traditional frameworks of grammar, such as dependency grammar \citep{tesniere:59} and LFG.\footnote{A motivation as to why these frameworks were chosen as reference points is provided in \S \ref{sec:frameworksofsyntax}.} The recognized set of grammatical functions in this work includes five elements: subject, object, oblique, predicate, and adjunct. Notably, the first four functions are categorized as arguments, assuming the argument-adjunct dichotomy.  This dichotomy, however, remains a controversial in the syntax literature (see \citealt{Vater1978, Haspelmath2014, prze:16}), primarily due to the absence of reliable tests for accurately classifying arguments and adjuncts. However, this thesis abstracts away from these debates, as the argument-adjunct dichotomy does not play a central role in the investigation. The hypothesis primarily operates on the functional labels of conjuncts and their (in)compatibility -- i.e., the hypothesis does not assert that argumenthood or adjuncthood plays a key role in Turkish coordination.
\end{sloppypar}

When defining and classifying grammatical functions, there is also no universally agreed-upon set of principles or tests, as languages employ diverse strategies to encode these functions. The present work, however, primarily relies on morphosyntactic tests for classifying grammatical functions. 

In the present study, the identification of the subject function relies primarily on case-marking\footnote{A brief overview of Turkish cases and their relation to grammatical functions is provided in \S\ref{sec:turkishcasesys}. For a more detailed overview, see \citet[pp.\ 45--58]{asli_kerslake_2010}.} (nominative in matrix clauses or genitive in subordinate clauses) and verbal agreement involving number and person. Distinguishing objects involves the passivization test as well as the identification of case-marking (typically accusative, but also nominative with non-specific objects). The oblique function in Turkish is realized by NPs that can assume different cases, typically the dative case. However, distinct from adjuncts, which can also be realized by NPs with different cases, the case of Turkish obliques is either determined by the governing verb or semantically assigned based on their thematic roles. When it comes to predicative arguments, they can be easily distinguished within the context of copular constructions. In Turkish, this copular item can be a suffix, such as \textit{-y-} marked directly on the predicative arguments, a separate verbal item like \textit{ol-} `be/become', or can be simply omitted, similar to Russian and Hebrew. The predicates themselves can be realized by AdjPs, PPs, and NPs in the nominative case. Consequently, the combined use of these morphosyntactic tests determines the classification of grammatical functions throughout the investigation into the coordination of unlikes.

\begin{sloppypar}
The hypothesis itself cannot be considered a novel one, as \citet{peterson2004} makes a similar claim for English coordination. However, to the best of this author's knowledge, neither the coordination of unlikes nor the specific hypothesis of the thesis have previously been tested against Turkish data. In this regard, the present thesis constitutes the first systematic attempt at investigating and analyzing unlike coordination within the domain of Turkish, a language natively spoken by around 82 million people \citep{turkish_ethnologue}.


The hypothesis is handled in four separate, yet inter-related dimensions in the body of the thesis: 1) Data exploration in annotated Turkish corpora; 2) Validation of the collected corpus data through a formal acceptability judgment survey filled out by native Turkish speakers; 3) Lexical-Functional Grammar analysis of Turkish coordination accounting for the validated empirical phenomena and native speaker intuitions; 4) Computational implementation of the proposed analysis in XLE (Xerox Linguistic Environment; \citealp{xle}). The first two dimensions not only form the empirical basis of the present research but also introduce novel coordination data to the ongoing debate. The remaining dimensions provide a formal foundation that can be used for further experimentation through parsing test suites and improving the existing large-scale, computational grammars of Turkish, such as the one developed by \citet{cetinoglu2009} in XLE.
\end{sloppypar}