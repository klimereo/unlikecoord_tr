\begin{sloppypar}
\begin{small}
Koordynacja to struktura gramatyczna, w której połączone są dwa lub więcej składników, zwykle za pomocą elementów zwanych spójnikami. Pomimo jej powszechności w językach naturalnych koordynacja pozostaje słabo zrozumiana. Jedna powszechnie akceptowana teoria zakłada, że gramatyczność koordynacji zależy od tego, czy skoordynowane elementy (koniunky) mają takie same cechy gramatyczne, przede wszystkim takie same przypadki i klasy morfosyntaktyczne. Chociaż teoria ta wielokrotnie była w przeszłości kwestionowana, debata wciąż koncentruje się na ograniczonych danych głównie z języka polskiego i angielskiego. W~celu pełniejszego podważenia założenia, że koniunkty muszą być składniowo podobne, niniejsze badanie wprowadza do debaty dwa rodzaje nowych danych empirycznych z języka tureckiego: autentyczne przykłady korpusowe i osądy rodzimych użytkowników języka.

Badanie zakłada, że gramatyczność koordynacji w~języku tureckim zależy od funkcji gramatycznej koniunktów, a nie od ich klas czy przypadków gramatycznych. Aby zbadać tę hipotezę, przeszukano duży korpus internetowy liczący 3,3 miliarda słów (Turkish Web 2012), używając odpowiednich zapytań w języku Corpus Query Language. W wyniku tego przeszukania uzyskano 137 przypadków koordynacji różnych klas oraz 51 przypadków koordynacji różnych przypadków. Warto zauważyć, że prawie wszystkie wydobyte przykłady (za wyjątkiem pięciu) zawierały koniunkty o zgodnych funkcjach gramatycznych.

Aby ocenić akceptowalność różnych konfiguracji koordynacji różnych klas i~przypadków, przeprowadzono formalne badanie akceptowalności z udziałem 48 rodzimych użytkowników języka tureckiego. Różne bodźce zdaniowe były oceniane na 7-stopniowej skali Likerta (od –3 do 3). Analiza wyników wskazuje, że niezgodność funkcji gramatycznych znacząco wpływa negatywnie na oceny akceptowalności, wspierając istotne znaczenie tożsamości funkcji gramatycznych – ale nie klas czy przypadków – w koordynacji.

Zaobserwowane fakty empiryczne zostały sformalizowane i zanalizowane w ramach teorii Lexical-Functional Grammar (LFG), zgodnie z podejściem zaproponowanym przez Przepiórkowskiego i Patejuk (2021). Aby zweryfikować analizę, przeprowadzono implementację obliczeniową przy użyciu środowiska Xerox Linguistic Environment, specjalistycznej platformy do implementacji gramatyk LFG.

Podsumowując, niniejsza praca dyplomowa przyczynia się do rosnącego zbioru badań kwestionujących powszechnie przyjmowaną tezę, że spójniki muszą być podobne pod względem przypadków i klas gramatycznych. % Ponadto implikacje tej pracy sięgają także do formalnej lingwistyki i inżynierii gramatycznej, oferując nowe perspektywy i możliwości dla przyszłych badań w tej dziedzinie.
\end{small}
\end{sloppypar}