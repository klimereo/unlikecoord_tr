\begin{sloppypar}
\begin{small}
Coordination is a grammatical structure where two or more linguistic elements are joined together, typically via elements called conjunctions. Despite its prevalence in natural languages, coordination remains poorly understood. One widely accepted view posits that the grammaticality of a coordinate structure depends on the coordinated elements (conjuncts) having the same grammatical properties, such as case and syntactic category. Although this view has been repeatedly questioned in the past, the debate still centers on limited data mainly from Polish and English. To further challenge the assumption that conjuncts must be alike, the present study introduces two types of novel empirical data from Turkish to the current debate: corpus examples and judgments of native speakers. 
	
The study hypothesizes that the grammaticality of coordinate structures in Turkish is contingent upon the grammatical function of the conjuncts, rather than their syntactic categories or cases. To investigate this hypothesis, a large web corpus of 3.3 billion words (Turkish Web 2012) was searched using tailored Corpus Query Language queries. This search yielded 137 instances of unlike category coordination and 51 instances of unlike case coordination. Notably, almost all extracted examples (with five exceptions) feature conjuncts with matching grammatical functions.
	
To assess the acceptability of different configurations of unlike category/case coordination, a formal acceptability judgment survey was conducted with 48 native speakers. Distinct sentence stimuli were rated on a 7-point Likert scale (from --3 to 3). An analysis of the results indicates that mismatching functions significantly affect the acceptability ratings, supporting the significance of grammatical function in coordination.
	
Finally, the observed empirical facts are formalized and analyzed within the framework of Lexical-Functional Grammar, adhering to the approach proposed by Przepiórkowski \& Patejuk (2021). To validate the analysis, a computational implementation was carried out using Xerox Linguistic Environment, a specialized platform for implementing grammars based on Lexical-Functional Grammar.
	
In conclusion, this thesis adds to the growing body of research challenging the widely assumed notion that conjuncts must be alike in their cases and categories. Moreover, the implications of this research extend to formal linguistics and grammar engineering.
\end{small}
\end{sloppypar}
